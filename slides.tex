\documentclass{beamer}

\mode<presentation>
{
  \usetheme{Warsaw}
  \usecolortheme{default}
  \usefonttheme{default}
  \setbeamertemplate{navigation symbols}{}
  \setbeamertemplate{caption}[numbered]
}

\usepackage[english]{babel}
\usepackage[utf8x]{inputenc}
\usepackage{listings}
\usepackage{amsmath}
\usepackage{amssymb}
 
\title[Putting Curry-Howard to Work]{Putting Curry-Howard to Work}
\author{Maciej Buszka}
\institute{Instytut Informatyki UWr}
\date{17.12.2018}

\lstset{language=Haskell}
\begin{document}


\begin{frame}
	\titlepage
\end{frame}


\begin{frame}{Curry-Howard vs programowanie}
  \begin{itemize}
    \item Izomorfizm Curry'ego-Howarda mówi nam o odpowiedniości typów i twierdzeń oraz dowodów i programów
    \item Jednak twierdzenia które dowodzimy pisząc w klasycznym języku programowania nie są specjalnie ciekawe
    \item Chcielibyśmy aby typy stanowiły twierdzenia o właściwościach programów które piszemy
    \item W tym celu musimy rozszerzyć język typów
  \end{itemize}
\end{frame}


\begin{frame}{Uwagi techniczne}
  \begin{itemize}
    \item Artykuł, na podstawie którego stworzona jest ta prezentacja pojawił się w 2005 roku.
    \item W tamtych czasach Haskell nie posiadał rozszerzeń opisanych w artykule.
    \item Dlatego autor (Tim Sheard) opisuje je w języku $\Omega$mega - podobnym do Haskella
    \item Aktualnie wszystkie rozszerzenia są dostępne w GHC (aczkolwiek niektóre mają inną implementację)
    \item Na tej prezentacji będę korzystał z Haskella, kompilowanego \texttt{GHC 8.6.x}
  \end{itemize}
\end{frame}


\section{System typów w Haskellu}
\subsection{Typy i wartości}
\begin{frame}[fragile]{Typy i wartości}
  \begin{itemize}
    \item W Haskellu mamy do dyspozycji typy wbudowane np. \lstinline!Int!, \lstinline!Char!
    \item Zdefiniowane w bibliotece standardowej np. \lstinline!String!, \lstinline![a]!
    \item Oraz zdefiniowane przez użytkownika:\\ \begin{lstlisting}
      data Person = Person 
        { name :: String
        , age  :: Int
        }
  
      data Maybe a 
        = Just a
        | Nothing
    \end{lstlisting}
  \end{itemize}
\end{frame}


\begin{frame}[fragile]{Typy i wartości}
  Definicja \lstinline!data Person = Person { ... }! wprowadza:
  \begin{itemize}
    \item Nową stałą typową \lstinline!Person!
    \item Nowy konstruktor \lstinline!Person :: String → Int → Person!
  \end{itemize}
  \vspace{1em}
  Natomiast \lstinline!data Maybe a = Just a | Nothing! wprowadza:
  \begin{itemize}
    \item Nowy konstruktor typów \lstinline!Maybe!
    \item Dwa konstruktory:
    \item \lstinline!Just :: a → Maybe a!
    \item \lstinline!Nothing :: Maybe a!
  \end{itemize}
\end{frame}


\subsection{Rodzaje i typy}
\begin{frame}[fragile]{Rodzaje i typy}
  \begin{itemize}
    \item Tak jak klasyfikujemy wartości za pomocą typów, typy są klasyfikowane poprzez rodzaje
    \item W standardowym Haskellu dostępna jest jedna stała rodzajowa \lstinline!Type! opisująca typy wartości np: \lstinline!Int :: Type!
    \item Oraz jeden konstruktor rodzajów \lstinline{ →}
    \item Przykładowo konstruktor typów może mieć rodzaj \lstinline!Maybe :: Type → Type!
    \item Nazwę \lstinline{Type} należy zaimportować z modułu \lstinline{Data.Kind}
    \item Kiedyś \lstinline{Type} nazywał się \lstinline{*}
  \end{itemize}
\end{frame}


\section{Rozszerzenia}
\subsection{Rodzaje definiowane przez użytkownika}
\begin{frame}[fragile]{Rodzaje definiowane przez użytkownika}
    \begin{itemize}
      \item Rozszerzenie \lstinline!DataKinds! pozwala na użycie definicji danych jako definicji nowego rodzaju
      \item Przykładowo definicja\\
      \begin{lstlisting}
        data Nat = Z | S Nat
      \end{lstlisting}
        wprowadza dodatkowo rodzaj \lstinline! 'Nat!, stałą typową \lstinline! 'Z :: 'Nat! oraz konstruktor typów \lstinline! 'S :: 'Nat → 'Nat!
      \item Jeżeli jest to jednoznaczne można opuścić \texttt{'}
      \item Warto zauważyć, że takie promowane typy nie klasyfikują żadnych wartości
    \end{itemize}
\end{frame}


\begin{frame}[fragile]{Klasyfikacje rodzajów}
  \begin{itemize}
    \item Gdy już mamy ciekawszy język typów i rodzajów pojawia się naturalne pytanie jak je sklasyfikować
    \item Jednym z podejść jest wybrane w artykule, polegające na konstrukcji hierarchii rodzajów:
    \begin{itemize}
      \item \lstinline{Int :: Type0}, \lstinline{Int → Bool :: Type0}, \lstinline{Maybe :: Type0 → Type0}
      \item \lstinline{Type0 :: Type1}, \lstinline{Type0 → Type0 :: Type1}
      \item rodzaj z poziomu $n+1$ klasyfikuje rzeczy z poziomu $n$
    \end{itemize} 
    \item Natomiast w Haskellu postanowiono dodać aksjomat \lstinline{Type :: Type}
  \end{itemize}
\end{frame}


\subsection{GADTs}
\begin{frame}[fragile]{GADTs}
  \begin{itemize}
    \item Rozszerzenie \lstinline!GADTs! pozwala na ogólniejsze definicje konstruktorów
    \item Korzystając z nowej składni możemy definiować algebraiczne typy danych jak wcześniej:
    \begin{lstlisting}
      data Maybe a where
        Just    :: a → Maybe a
        Nothing :: Maybe a
    \end{lstlisting}
  \end{itemize}
\end{frame}


\begin{frame}[fragile]{GADTs}
  \begin{itemize}
    \item Rozszerzenie \lstinline!GADTs! pozwala na ogólniejsze definicje konstruktorów
    \item Ale także:
    \begin{lstlisting}
data IntOrBool a where
  AnInt :: IntOrBool Int
  ABool :: IntOrBool Bool
    \end{lstlisting}
    \item Takie ukonkretnienie typu jest widoczne podczas destrukcji:
  \end{itemize}
\begin{lstlisting}
check :: IntOrBool a → String
check (AnInt i) = "An Int " ++ show (i + 42)
check (ABool b) = "A Bool " ++ show (not b) 
\end{lstlisting}
\end{frame}


\begin{frame}[fragile]{GADTs}
  Definicję postaci:
  \begin{lstlisting}
    data IntOrBool a where
      AnInt :: IntOrBool Int
      ABool :: IntOrBool Bool
  \end{lstlisting}
  możemy równoważnie przepisać jako:
  \begin{lstlisting}[mathescape]
    data IntOrBool a where
      AnInt :: (a ~ Int)  $\Rightarrow$ IntOrBool a
      ABool :: (a ~ Bool) $\Rightarrow$ IntOrBool a
  \end{lstlisting}
  gdzie \lstinline{~} wprowadza nowe ograniczenie do rozwiązania przez system typów.
\end{frame}


\subsection{Rodziny typów}
\begin{frame}[fragile]{Rodziny typów}
  \begin{itemize}
    \item Rozszerzenie \lstinline{TypeFamilies} pozwala nam na definicje funkcji na poziomie typów
    \item Przykładowo:
\begin{lstlisting}
type family Add n m :: Nat where
  Add Z     m = m
  Add (S n) m = S (Add n m)
\end{lstlisting}
definiuje funkcję dodającą do siebie dwie liczby naturalne zakodowane jako rodzaj \lstinline!Nat!
  \end{itemize}
\end{frame}


\section{Przykłady i wzorce}
\begin{frame}{Przykłady i wzorce}
  \begin{itemize}
    \item przykłady z artykułu, przetłumaczone na Haskella
    \item \href{https://github.com/goldfirere/glambda}{GLambda} autorstwa Richarda Eisenberga
    \item \href{https://github.com/sweirich/dth}{Dependently typed Haskell} autorstwa Stephanie Weirich
    \item \href{https://github.com/goldfirere/singletons}{biblioteka singletons} pozwalająca na automatyczne generowanie singletonów
    \item Kod z artykułów Justina Le
  \end{itemize}
\end{frame}

\begin{frame}{Bibliografia}
  \begin{thebibliography}{10}
		\setbeamertemplate{bibliography item}[article]
  	\bibitem{curryHoward}
    	Tim Sheard
    	\newblock {\em Putting Curry-Howard to Work}
		\setbeamertemplate{bibliography item}[article]
		\bibitem{singletonsPaper}
			Richard A. Eisenberg
			\newblock {\em Dependently Typed Programming with Singletons}.
		\setbeamertemplate{bibliography item}[online]
  	\bibitem{introToSingletons}
    	Justin Le
      \newblock {\em \href{https://blog.jle.im/entries/series/+introduction-to-singletons.html}{Introduction to Singletons}}.
    \setbeamertemplate{bibliography item}[online]
      \bibitem{glambda}
        \href{https://github.com/goldfirere/glambda}{GLambda}
    \setbeamertemplate{bibliography item}[online]
      \bibitem{dth}
        \href{https://github.com/sweirich/dth}{Dependently typed Haskell}
    \setbeamertemplate{bibliography item}[online]
      \bibitem{singletonsLib}
        \href{https://github.com/goldfirere/singletons}{singletons}
  \end{thebibliography}
\end{frame}

\end{document}
